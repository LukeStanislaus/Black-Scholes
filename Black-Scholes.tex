% !TEX TS-program = pdflatex
% !TEX encoding = UTF-8 Unicode

% This is a simple template for a LaTeX document using the "article" class.
% See "book", "report", "letter" for other types of document.

\documentclass[12pt]{article} % use larger type; default would be 10pt

\usepackage[utf8]{inputenc} % set input encoding (not needed with XeLaTeX)
\usepackage{biblatex} %Imports biblatex package
\addbibresource{Black-Scholes.bib} %Import the bibliography file

%%% Examples of Article customizations
% These packages are optional, depending whether you want the features they provide.
% See the LaTeX Companion or other references for full infofrmation.

%%% PAGE DIMENSIONS
\usepackage{geometry} % to change the page dimensions
\geometry{a4paper} % or letterpaper (US) or a5paper or....
% \geometry{margin=2in} % for example, change the margins to 2 inches all round
% \geometry{landscape} % set up the page for landscape
%   read geometry.pdf for detailed page layout information

\usepackage{graphicx} % support the \includegraphics command and options

% \usepackage[parfill]{parskip} % Activate to begin paragraphs with an empty line rather than an indent

%%% PACKAGES
%\usepackage{hyperef}
\usepackage{booktabs} % for much better looking tables
\usepackage{amsmath}
\usepackage{array} % for better arrays (eg matrices) in maths
\usepackage{paralist} % very flexible & customisable lists (eg. enumerate/itemize, etc.)
\usepackage{verbatim} % adds environment for commenting out blocks of text & for better verbatim
\usepackage{subfig} % make it possible to include more than one captioned figure/table in a single float
% These packages are all incorporated in the memoir class to one degree or another...

%%% HEADERS & FOOTERS
\usepackage{fancyhdr} % This should be set AFTER setting up the page geometry
\pagestyle{fancy} % options: empty , plain , fancy
\renewcommand{\headrulewidth}{0pt} % customise the layout...
\lhead{}\chead{}\rhead{}
\lfoot{}\cfoot{\thepage}\rfoot{}

%%% SECTION TITLE APPEARANCE
\usepackage{sectsty}
\allsectionsfont{\sffamily\mdseries\upshape} % (See the fntguide.pdf for font help)
% (This matches ConTeXt defaults)

%%% ToC (table of contents) APPEARANCE

\usepackage{amsfonts} 
\usepackage[nottoc,notlof,notlot]{tocbibind} % Put the bibliography in the ToC
\usepackage[titles,subfigure]{tocloft} % Alter the style of the Table of Contents
\renewcommand{\cftsecfont}{\rmfamily\mdseries\upshape}
\renewcommand{\cftsecpagefont}{\rmfamily\mdseries\upshape} % No bold!
%%% END Article customizations

\newtheorem{theorem}{Theorem}[section]
\newtheorem{corollary}{Corollary}[theorem]
\newtheorem{lemma}[theorem]{Lemma}


%%% The "real" document content comes below...

\title{GameStop: A study in the Black-Scholes Formula}
\author{Luke Stanislaus 2009701}
%\date{} % Activate to display a given date or no date (if empty),
         % otherwise the current date is printed 

\begin{document}

\maketitle

\begin{abstract}
    This is a simple paragraph at the beginning of the document. A brief introduction to the main subject.
    \end{abstract}

\section{Introduction}

In the first section of my essay, I plan to outline what happened in February with
GameStop and the internet, showing graphs of the situation, for example 
%\href{https://static01.nyt.com/images/2021/02/02/us/gamestop-stock-promo-1612301679434/gamestop-stock-promo-1612301679434-videoSixteenByNineJumbo1600.png}{the stock price over time} 
%and the \href{https://www.ft.com/__origami/service/image/v2/images/raw/https%3A%2F%2Fd6c748xw2pzm8.cloudfront.net%2Fprod%2F034d8f30-6237-11eb-a858-6b1fccb4dc6e-standard.png?dpr=1&fit=scale-down&quality=highest&source=next&width=700}{short interest} 
in GameStop, in order to motivate the reader in our analysis of the Black-Scholes formula
under the bizarre situation of GameStop. I can also talk about similar situations 
in the past of similar significant "short squeezes", and consider how analysing 
GameStop could allow us to predict future short squeezes with different companies by 
analysing their financial situation. 
\paragraph
I would then move on to explaining the Black-Scholes Model, and could attempt a simple 
derivation of the Black-Scholes Formula. I can then move on to solving the 
formula for different intial conditions and variables, using my tools from 
Programming for Scientists last year to generate graphs of the predicted stock value 
(and value of options) in Python, where suitable. I will then take a conclusion from 
my calculations, deciding whether the prelevence of GameStop on messageboards and 
the internet is what caused the large spike, or if the high short interest in GameStop 
meant that it was inevitable to happen anyway.
\paragraph
I would expect to use Jupyter notebook and Python for graph sketching, and I plan to 
reference "Stochastic Differential Equations: An Introduction with Applications" by 
Bernt Karsten Øksendal, and "Options, Futures and Other Derivatives" by John C. Hull 
as part of writing my essay, as they contain critical information relevant to my topic.
Below is the Black-Scholes formula:

$\frac{\partial V}{\partial t} + \frac{1}{2}\sigma^2 S^2 \frac{\partial^2 V}{\partial S^2} 
+ rS\frac{\partial V}{\partial S} - rV = 0$
Where t is time in years, r is the interest rate, S is the price of the underlying asset 
and $\sigma$ is the standard deviation of the stock's returns.
\paragraph
I can also talk about other, more complicated financial models and how they compared to 
Black-Scholes, for example which assumptions they start to calculate which are ignored
in Black-Scholes.

$\ A(t) = rA(t) $

Let's cite! Einstein's journal paper \cite{einstein} and Dirac's
book \cite{dirac} are physics-related items. 

\section{What are shares, options, and short selling?}

\section{What happened to GameStop?}

\section{Asset dynamics}

\subsection{A risk-free asset}
A risk-free asset, as defined in The Black-Scholes Model \cite{blackscholes} can be thought
of as a money-market account with zero risk, for example a bank account with an interest 
rate. It is described by the deterministic function 
\begin{equation} \label{riskfree}
    dA(t) = rA(t)\mathrm{d}t 
\end{equation}
Where in \ref{riskfree}, $r$ represents the risk-free rate and $A$ the value of the 
asset. We typically let $r>0$ and $A(0) = 1$ for convenience. This can be rewritten as 
the ordinary differential equation $A'(t) = rA(t)$, which has the unique solution 
\begin{equation} \label{riskfree solution}
    A(t) =e^{rt}
\end{equation}
\subsection{Ito processes}

According to Ito process \cite{itoprocess}, an Ito process is the stochastic (random) 
process $X = \{X_t, t>=0\}$ which solves 
\begin{equation} \label{randomprocess}
X_t = X_0 + \int_{0}^{t} a(X_s, s)\mathrm{d}s + 
\int_{0}^{t} b(X_s, s) \mathrm{d}W_s
\end{equation}
Here $X_0$ is the "scalar starting point", and $a,b$ are stochastic 
processes defined for $\{a(X_t, t) : t\geq0\} $ and $\{b(X_t, t) : t\geq0\}$. 
Then we call $a$ the drift of the variable, and $b$ is the diffusion, 
or more precisely in a financial context, volatity. \ref{randomprocess} is commonly 
written as 
\begin{equation}
    X_t = a(X_t, t)\mathrm{d}t + b(X_t, t)\mathrm{d}W_t
\end{equation}
or even 
\begin{equation}
    X_t = a_t\mathrm{d}t + b_t\mathrm{d}W_t
\end{equation}
This converges at some time $T$ to Brownian motion with instantaneous drift $a_T$ and
variance $b_T^2$. We let $dW$ be normally distributed with mean zero and variance 
$\mathrm{d}t$. Then we can further simplify \ref{randomprocess} as 
\begin{equation}
    \mathrm{d}X_t = a_t\mathrm{d}t + b_t \sqrt{\mathrm{d}t}\xi
\end{equation}
where
\begin{equation}
    \xi \sim \mathcal{N}(\mu,\,\sigma^{2})
\end{equation}
As defined in \cite{itoprocess}, Ito's lemma is 

\begin{lemma}[Ito's Lemma]
    Suppose $f: R \to R$ is twice continuously differentiable and 
    $\mathrm{d}X = a_t\mathrm{d}t + b_t\mathrm{d}W$. Then $f(X)$ is the Ito process,
    \begin{equation}
        f(X_t)
        = f(X_0) + \int_0^t \! f'(X_s)a_s \, \mathrm{d}s + 
        \int_0^t \! f'(X_s)b_s \, \mathrm{d}W + \frac{1}{2}\int_0^t \! f''(X_s)b_s^2 \,
         \mathrm{d}s
    \end{equation}
    for $t\ge0$
\end{lemma}

\subsection{A risky asset}

A risk asset, as defined in The Black-Scholes Model \cite{blackscholes}, 
can be thought of as a stock, is represented as the Ito process

\begin{equation}
    \mathrm{d}S(t) = \mu S(t)\mathrm{d}t + \sigma S(t) \mathrm{d}W(t)
\end{equation}

with $S(0)$ a given starting price of the stock, $\mu \in \mathbb{R}$ 
is again the drift and $\sigma > 0$ is the volatility of the stock price $S$.
\section{Options}

\section{The Black-Scholes Model}

\section{Modelling GameStop}

\section{Extensions to the Black-Scholes Model}

\section{Conclusion}

\printbibliography %Prints bibliography
\end{document}
