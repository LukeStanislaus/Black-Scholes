% !TEX TS-program = pdflatex
% !TEX encoding = UTF-8 Unicode

% This is a simple template for a LaTeX document using the "article" class.
% See "book", "report", "letter" for other types of document.

\documentclass[11pt]{article} % use larger type; default would be 10pt

\usepackage[utf8]{inputenc} % set input encoding (not needed with XeLaTeX)
\usepackage{biblatex} %Imports biblatex package
\addbibresource{Black-Scholes.bib} %Import the bibliography file
\usepackage{graphicx}
\graphicspath{ {./images/} }

%%% Examples of Article customizations
% These packages are optional, depending whether you want the features they provide.
% See the LaTeX Companion or other references for full infofrmation.

%%% PAGE DIMENSIONS
\usepackage{geometry} % to change the page dimensions
\geometry{a4paper} % or letterpaper (US) or a5paper or....
% \geometry{margin=2in} % for example, change the margins to 2 inches all round
% \geometry{landscape} % set up the page for landscape
%   read geometry.pdf for detailed page layout information

\usepackage{graphicx} % support the \includegraphics command and options

% \usepackage[parfill]{parskip} % Activate to begin paragraphs with an empty line rather than an indent

%%% PACKAGES
%\usepackage{hyperef}
\usepackage{booktabs} % for much better looking tables
\usepackage{amsmath}
\usepackage{array} % for better arrays (eg matrices) in maths
\usepackage{paralist} % very flexible & customisable lists (eg. enumerate/itemize, etc.)
\usepackage{verbatim} % adds environment for commenting out blocks of text & for better verbatim
\usepackage{subfig} % make it possible to include more than one captioned figure/table in a single float
% These packages are all incorporated in the memoir class to one degree or another...

%%% HEADERS & FOOTERS
\usepackage{fancyhdr} % This should be set AFTER setting up the page geometry
\pagestyle{fancy} % options: empty , plain , fancy
\renewcommand{\headrulewidth}{0pt} % customise the layout...
\lhead{}\chead{}\rhead{}
\lfoot{}\cfoot{\thepage}\rfoot{}

%%% SECTION TITLE APPEARANCE
\usepackage{sectsty}
\allsectionsfont{\sffamily\mdseries\upshape} % (See the fntguide.pdf for font help)
% (This matches ConTeXt defaults)

%%% ToC (table of contents) APPEARANCE

\usepackage{amsfonts} 
\usepackage[nottoc,notlof,notlot]{tocbibind} % Put the bibliography in the ToC
\usepackage[titles,subfigure]{tocloft} % Alter the style of the Table of Contents
\renewcommand{\cftsecfont}{\rmfamily\mdseries\upshape}
\renewcommand{\cftsecpagefont}{\rmfamily\mdseries\upshape} % No bold!
%%% END Article customizations

\newtheorem{theorem}{Theorem}[section]
\newtheorem{corollary}{Corollary}[theorem]
\newtheorem{lemma}[theorem]{Lemma}


%%% The "real" document content comes below...

\title{GameStop: A study in the Black-Scholes Formula}
\author{Luke Stanislaus 2009701}
%\date{} % Activate to display a given date or no date (if empty),
         % otherwise the current date is printed 

\begin{document}

\maketitle

\begin{abstract}
    This essay is about the Black-Scholes formula, and how it could be 
    used to explain the short squeeze of GameStop in early 2021. We begin 
    with explaining shares, options and short selling, and move on to 
    deriving the Black-Scholes formula, and finally apply it to GameStop.
    \end{abstract}

%\section{Introduction}


\section{What are shares, options, and short selling?}
\subsection{Shares}
As defined in "Investopedia" \cite{shares}, shares are "units of equity 
ownership in a corporation." Writing shares is an effective way to raise 
capital for a firm, as there is not legal mandate to be repaid to 
investors, and to not pay interest. Common shares usually offer a voting 
rights, giving shareholders more control over a business. The value of 
shares rises and falls with demand - meaning that owning a stock assumes 
the owner a level of risk. Many stocks also offer dividends to 
shareholders, which is a proportion of the businesses profits paid out 
directly.
\subsection{Options}
Again, as defined in \cite{options}, "the term option refers to a 
financial instrument based on the value of the underlying security, 
such as stocks." An options contract offers the holder the option, 
not the obligation (in contrast to a "future"), to buy or sell a share 
at a predetermined price. In themselves, they are a form of asset and 
as such have their own valuations and market value. A "call" option 
is when the writer of the option promises to sell the stock at a 
predetermined price, called the "strike" at a predetermined time, 
called the "strike date". This means that if you own a call option and 
the stock price increases above the strike, the writer will still have 
to honor the strike and so you can "exercise" the option 
and immediately sell the stock, where your profit is the difference 
between the strike and current price, less the cost of the option. 
However, if the price goes below the strike, you would be forced 
to let your option expire worthless, since by exercising the option 
you would pay above the market rate. In this way, the losses are limited 
but the profits are unlimited - you can only lose what you paid for the 
call options, but stand to gain an unlimited amount as the stock price 
increases. 
\paragraph{}
On the other hand, if an investor expects a stock to decrease 
in value, they could write a call (also known as a short call). This 
means that if the stock has increase in value and you don't already own 
the security, you are obligated to purchase at the market rate, which 
could take any arbitrary value. Further to this, as the price increases 
over time, the short seller may be "margin called" - when to ensure the 
seller can pay, the broker requests that the seller puts more money into 
their account. If the short seller cannot afford this (and the further 
risk it entails), they can close their position - by buying the stocks 
they owe, the are protected from any further increases in price. However, 
by buying the stock off the open market they increase demand for the 
stock, increasing the price further, forcing other short sellers to 
close their positions, driving the price up further. This is what is 
known as a short squeeze.
\paragraph{}\label{typesofoption}
There are also two types of options - a "european" and "american" 
derivative security. An european option can only be exercised at the 
strike date, while an american option can be exercised at any time up 
to the strike date. In this essay we shall only consider the simpler 
european option, although the american option is far more common in 
the financial markets.

\section{Short squeezes and GameStop}
A short squeeze is relatively uncommon occurance in financial markets, but they do 
happen, normally due to substancial short interest in the stop, i.e. a large number 
of the shares sold short, an a samll available float - the number of shares 
available to purchaase on the market. This means, as the short sellers (traders 
writing call options) are forced to close their positions, they are fighting over 
a small number of stocks, inflating the price to extraordinary values. 
\paragraph{}
For example, 
in 2008, Porsche began to buy large amounts of Volkswagen stock off the market. This 
meant that only 6\% of the stock was available on the market, while 13\% of the stock 
was sold short. Th\begin{figure}[h]
    \centering
    \includegraphics[width=0.5\textwidth]{volkswagen.jpg}
    \caption{The stock price of Volkswagen during the short squeeze of 2008}
    \end{figure}is caused a scramble for the final few shares, and as short sellers 
were squeezed out by the high price and were forced to close their position by buying 
more of the stock, the price went higher and higher, and Volkswagen briefly became 
the highest value company in the world. The squeeze only ended when Porsche announced 
it would sell around 5\% of its stake in order to make life easier for the hedge 
funds - it is estimated that the squeeze cost short sellers around £30 billion, 
according to \cite{volkswagen}, with massvive profits going over to Porsche.

\paragraph{}
The case of GameStop is similar - there was an extreme amount of short interest in 
the share, so that at one point more than \%100 of the stock was sold short. With 
\begin{figure}[h]
    \centering
    \includegraphics[width=0.75\textwidth]{gamestopprice.png}
    \caption{The stock price of GameStop over the past 5 years.}\label{gamestopprice}
    \end{figure}
    COVID and the death of the high street, a brick and mortar store like GameStop was 
considered an extremely safe short sell - and as you can see from \ref{gamestopprice}  
until early 2021, the price of GameStop was dropping extremely consistently. This  
meant that it had an extremely high short interest, as you can see in \ref{shortinterest}. 
This was noticed by a messageboard on the internet, called "r/wallstreetbets" \cite{wsb}. 
\begin{figure}[h]
    \centering
    \includegraphics[width=0.5\textwidth]{shortinterest.png}
    \caption{The short interest in GameStop over the past 5 years.}
    \label{shortinterest}
    \end{figure}
The independant investors saw 
this as greed from the large hedge funds making billions of safe money from the demise 
of a high street store, and when Ryan Cohen, who had just created the hugely sucessful 
"Chewy", the largest onine pet retailer, joined GameStop's board as chairman. Investors
on r/wallstreetbets saw this as a very positive sign, as they imagined Ryan could turn 
GameStop into "Amazon, but for computer games". Hence they attempted to create a 
short squeeze in GameStop, hoping to profit from the greed of the hedge funds. They 
began purchasing the stock, both increasing the price and reducing the available float, 
so that as short sellers attempted to close their positions due to rising prices, they 
found themselves fighting over a smaller and smaller number of shares, similar to the 
Volkswagen case. This caused a signfiicant short squeeze in the price, profiting the 
independant investors who had invested early on, as seen in \ref{shortsqeeze}.
\begin{figure}[h]
    \centering
    \includegraphics[width=0.5\textwidth]{shortsqueeze.png}
    \caption{The stock price of GameStop in early 2021, during the squeeze.}
    \label{shortsqeeze}
    \end{figure}

\section{Asset dynamics}\label{assetdyamics}
However, the squeeze was short lived - as the stock increased tenfold, independant 
investors began to cash out on their profits and sell their stock, the pressure 
on the short sellers became reduced and they were able to exit their positions. 
This is in contrast to the Volkswagen case, where the shares were all owned by one 
party, who completely refused to sell its shares, pushing the share price much higher.
\subsection{A risk-free asset}
A risk-free asset, as defined in The Black-Scholes Model \cite{blackscholes} can be thought
of as a money-market account with zero risk, for example a bank account with an interest 
rate. It is described by the deterministic function 
\begin{equation} \label{riskfree}
    dA(t) = rA(t)\mathrm{d}t 
\end{equation}
Where in \ref{riskfree}, $r$ represents the risk-free rate and $A$ the value of the 
asset. We typically let $r>0$ and $A(0) = 1$ for convenience. This can be rewritten as 
the ordinary differential equation $A'(t) = rA(t)$, which has the unique solution 
\begin{equation} \label{riskfree solution}
    A(t) =e^{rt}
\end{equation}
\subsection{Ito processes}

According to Ito process \cite{itoprocess}, an Ito process is the stochastic (random) 
process $X = \{X_t, t>=0\}$ which solves 
\begin{equation} \label{randomprocess}
X_t = X_0 + \int_{0}^{t} a(X_s, s)\mathrm{d}s + 
\int_{0}^{t} b(X_s, s) \mathrm{d}W_s
\end{equation}
Here $X_0$ is the "scalar starting point", and $a,b$ are stochastic 
processes defined for $\{a(X_t, t) : t\geq0\} $ and $\{b(X_t, t) : t\geq0\}$. 
Then we call $a$ the drift of the variable, and $b$ is the diffusion, 
or more precisely in a financial context, volatity. \ref{randomprocess} is commonly 
written as 
\begin{equation}
    X_t = a(X_t, t)\mathrm{d}t + b(X_t, t)\mathrm{d}W_t
\end{equation}
or even 
\begin{equation}
    X_t = a_t\mathrm{d}t + b_t\mathrm{d}W_t
\end{equation}
This converges at some time $T$ to Brownian motion with instantaneous drift $a_T$ and
variance $b_T^2$. We let $dW$ be normally distributed with mean zero and variance 
$\mathrm{d}t$. Then we can further simplify \ref{randomprocess} as 
\begin{equation}
    \mathrm{d}X_t = a_t\mathrm{d}t + b_t \sqrt{\mathrm{d}t}\xi
\end{equation}
where
\begin{equation}
    \xi \sim \mathcal{N}(\mu,\,\sigma^{2})
\end{equation}

As defined in \cite{itoprocess}, Ito's lemma is 

\begin{lemma}[Ito's Lemma] 
    \label{itolemma}
    Suppose $f: R \to R$ is twice continuously differentiable and 
    $\mathrm{d}X = a_t\mathrm{d}t + b_t\mathrm{d}W$. Then $f(X)$ is the Ito process,
    \begin{equation}
        f(X_t)
        = f(X_0) + \int_0^t \! f'(X_s)a_s \, \mathrm{d}s + 
        \int_0^t \! f'(X_s)b_s \, \mathrm{d}W + \frac{1}{2}\int_0^t \! f''(X_s)b_s^2 \,
         \mathrm{d}s
    \end{equation}
    for $t\ge0$
\end{lemma}
We shall also require Ito's existence/uniqueness theorem, as defined in 
\cite{SDE}:

\begin{theorem}[Ito's Existence And Uniqueness Theorem] 
    \label{isosexistence}
    If $\mu : \mathbb{R} \to \mathbb{R}$ and $\sigma : \mathbb{R} \to \mathbb{R}_+$ 
    are uniformally Lipschitz, then the stochastic differential equation 
    \ref{randomprocess} has "strong solutions". This means that for any standard 
    Brownian motion $\{W_t\}_{t\geq0}$, any admissable filtration $\mathbb{F} = 
    \{\mathcal{F}_t\}$ and any initial value $x \in \mathbb{R}$ there exists a 
    unique proccess $X_t = X_t^x$ which solves \ref{randomprocess}.

    
\end{theorem}

\subsection{A risky asset}

A risk asset, as defined in The Black-Scholes Model \cite{blackscholes}, 
can be thought of as a stock, is represented as the Ito process

\begin{equation} \label{riskyasset}
    \mathrm{d}S(t) = \mu S(t)\mathrm{d}t + \sigma S(t) \mathrm{d}W(t)
\end{equation}

with $S(0)$ a given starting price of the stock, $\mu \in \mathbb{R}$ 
is again the drift and $\sigma > 0$ is the volatility of the stock 
price $S$. Using \ref{isosexistence} we can prove existence and uniqueness of 
\ref{riskyasset}:

First we rewrite \ref{riskyasset} in integral form:
\begin{equation}\label{riskyint}
    S(t) = S(0) + \mu\int_0^t \! S(u) \, \mathrm{d}u + \sigma\int_0^t \! S(u) \, 
    \mathrm{d}W(u).
\end{equation}

Then we rewrite our coefficients in the form of \ref{randomprocess}: 

\begin{align}
    \mu S(t) = a(t,S(t)), && a(t,x) = \mu x \\
    \sigma S(t) = b(t,S(t), && b(t,x) = \sigma x
\end{align}

then we check Lipschitz continuity:

\begin{align}
    |a(t,x) - a(t,y)| &= |\mu(x-y)| \leq |\mu||x-y|, \\
    |b(t,x) - b(t,y)| &= |\sigma(x-y)| \leq \sigma|x-y| &&\text{Since $\sigma > 0$.}
\end{align}

We then have that the unique solution to \ref{riskyasset} is of the form 
\begin{equation}
    S(t) = S(0)exp\{\mu t - \frac{\sigma^2}{2}t + \sigma W(t)\}.
\end{equation}
This can be easily substituted into \ref{riskyint} show that this solves \ref{riskyasset}.

\subsection{Considering the model parameters}

Now we have a soluition to a risky asset, we can consider $\mathbb{E}(S(t))$:

\begin{align}
    \mathbb{E}(S(t))  = &&S(t)\mathbb{E}(exp\{\mu - \frac{1}{2}\sigma t + \sigma W(t)\})\\
     \label{variance}= 
     && S(0)exp\{\mu t - \frac{1}{2}\sigma^2 t\}\mathbb{E}(exp\{\sigma W(t)\} \\
    \label{result} = && S(0)exp\{\mu t\}).
\end{align}

Where at \ref{variance} we use the fact that 

\begin{equation*}
    \mathbb{E}(exp\{X\}) = exp\{\frac{1}{2}Var(X)\}
\end{equation*}

\ref{result} then shows that if $\mu = 0$ we know that the expectation of S(t) is 
constant in time - it is not "drifting" in any direction, which justifies the 
naming of $\mu$ as drift. We can then rearrange for $\mu$:

\begin{equation} \label{mu}
    \mu = \frac{1}{t}\ln{\frac{\mathbb{E}(S(t))}{S(0)}}
\end{equation}

which is the logarithmic return of the expected price.
The variance of the return is 
\begin{align}
    Var(\mu t - \frac{\sigma^2}{2}t + \sigma W(t)) = && Var(\sigma W(t)) \\
    = && \sigma^2t  && \text{since Var(W(t)) = t}
\end{align}
as defined in \cite{blackscholes}. Clearly we want to try to calculate $\mu$ and 
$\sigma$ to apply to our model. \ref{mu} could suggest taking past values of the 
stock value and taking an average to find a value of $\mu$, but according to 
\cite{blackscholes} the accuracy of this is poor. 
A more effective approximation of the volatility is, for example, the process 

\begin{equation}
    \ln{S(t)} = \ln{S(0)} + (\mu - \frac{1}{2}\sigma^2)t + \sigma W(t))
\end{equation}

which is an Ito process with the constant characteristics $a(t) = (\mu - \frac{1}{2}\sigma^2)$
 and $b(t) = \sigma$. By \cite{quadtraticvariation}, we have that
 \begin{equation}
     (\mathrm{d}S(t))^2 = b^2(t) \mathrm{d}t
 \end{equation}
 If we then partition $[0,t]$ given by $0 = t_1 < \dots < t_n = t$ with small mesh max 
 width $[t_{k+1} - t_k]$, we have 
 \begin{equation}
     (\ln{S(t_{k+1}) - \ln{S(t_k)}})^2 \approx \sigma^2 \mathrm{d}t
 \end{equation}
then 
\begin{align}
    \sum_k (\ln{S(t_{k+1}) - \ln{S(t_k)}})^2 \approx \sigma^2 t\\
    \implies
    \sigma = \sqrt{\frac{1}{2}\sum_k (\ln{\frac{S(t_{k+1})}{S(t_k)}})^2}
    \label{volatilityapprox}
\end{align}
which is a much better estimate of the volatility coefficient, and is described as the 
sample volatility.
\section{Options}
\subsection{Assumptions}

As in \cite{blackscholes}, before we can begin to derive the Black-Scholes Model, we must 
state some assumptions.
\subsubsection{Existence of replicating strategy}\label{replicatingstrategy}

This assumption says that any option investment can be perfectly replciated by holdings in 
the stock and a money market account directly. A key part of investing is the substantial 
"leverage" employed in an option, as for a small (as a percentage of the share value) cost 
you can purchase a large number of shares and sell them for a profit, without having to risk 
investing such a large amount of money directly into the stock. Our assumption, however, says 
that a "replicating strategy" always exists. More rigorously, we say that there exists a 
pair of processes $(x,y)$ which satisfy 
\begin{equation}
    H(t) = x(t)S(t) + y(t)A(t)
\end{equation}
where $H(t)$ is the option value at time T, an Ito proces.$S(t)$ and $A(t)$ are the risky 
and riskless assets as defined in Asset Dynamics \ref{assetdyamics}. 
As described in \cite{blackscholes}, this assumption "captures the idea that changes in the 
values and holdings of assets are sole drivers of changes of wealth".

Additionally we assume that the process $H(t)$ is of  the form 
\begin{equation}
    H(t) = u(t, S(t)).
\end{equation}

This deterministic function $u(t,z)$ is not dependant on the history of the stock price. 
It is assmued to have continuous first derivative wrt $x \in [0,T]$ and continuous first 
and second derivatives in $z \in \mathbb{R}$.

\subsubsection{Assumption 2}
There again exists a replicating strategy $(x,y)$ which satisfies 
\begin{equation}\label{replicatingderivative}
    \mathrm{d} H(t) = x(t)\mathrm{d} S(t) + y(t)\mathrm{d} A(t)
\end{equation}
where $H$, $S$, $x$, $y$ and $A$ are as defined in \ref{replicatingstrategy}.
\subsubsection{Assumption 3}
There exists a probability Q


Applying \ref{itolemma} with $u(t,Z) = f(Z)$ we get   
\begin{align}\label{itoformulasolve}
    \mathrm{d}u(Z)  = u_z(Z)\mu \mathrm{d}t + u_z(Z)\sigma \mathrm{d}W + \frac{1}{2}
    u_{zz}(Z) \sigma^2 \mathrm{d}t&& \text{and} \\
    \mathrm{d}u(t) = u_t &&\text{then} \\
    \mathrm{d}H =\mathrm{d}u = \mathrm{d}u(Z) +\mathrm{d}u(t) =   (u_t + \mu S u_z + \frac{1}{2} 
    \sigma^2 S^2 u_{zz})\mathrm{d}t + +\sigma S u_z \mathrm{d}W
\end{align}
as a stochastic differential.
From \ref{replicatingderivative}, we have
\begin{align}
    \mathrm{d}H = x(t)\mathrm{d}S(t) +\mathrm{d}x(t)S(t) 
\end{align}
so
\begin{equation}\label{financingcondition}
    \mathrm{d}H = (x\mu S + ryA)\mathrm{d}t + x \sigma S \mathrm{d}W 
\end{equation}

from \ref{riskfree} and \ref{riskyasset}.

Then we equate the right hand side of \ref{itoformulasolve} and \ref{financingcondition}:

\begin{align}\label{dt}
    u_t + \mu S u_z + \frac{1}{2} \sigma^2 S^2 u_{zz}  = &&x\mu S + ryA &&\text{and} \\
    \sigma Su_z =&& x\sigma S.\label{dw}
\end{align}

Now we can start to solve these equations: \ref{dw} gives us
\begin{equation}
    x(t) = u_z(t,S(t))
\end{equation}

which we can use to eliminate x from \ref{dt}:

\begin{equation}
    u_t + \frac{1}{2}\sigma ^2 S^2u_{zz} = ryA
\end{equation}

which we then may use to solve for $y(t)$:

\begin{equation}
    y(t) = \frac{1}{rA(t)}(u_t(t,S(t)) + \frac{1}{2} \sigma^2S^2(t)u_{zz}(t,S(t)) ).
\end{equation}
where I have reintroduced the arguements for all the functions.

Finally, we use \ref{replicatingstrategy} and our expressions for $x$ and $y$ to give us
\begin{align}
    u(t,S(t)) = u_z(t,S(t))S(t) + \frac{1}{r}\bigg(u_t(t,S(t) + \frac{1}{2}\sigma^2S^2(t)
    u_{zz}(t,S(t))\bigg) \\\text{since $H(t) = u(t,S(t))$}
\end{align}
We now simply rearrange for $u_t(t,z)$:
\begin{align}\label{blackscholes}
    u_t(t,z) = -\frac{1}{2}\sigma^2z^2u_{zz}(t,z) - rzu_z(t,z) + ru(t,z) 
   && \text{for $0<t<T$, $z \in \mathbb{R}$.}
\end{align}
We also have the boundary condition that $H(T)$ is equal to the option payoff at the strike 
date, so in the case of a call option we can immediately buy the stock to lock in our profits, 
so a $t=0$ we have
\begin{align} \label{initialvalue}
    u(0,z) = \max{(z - K , 0)} && \text{for $z \in \mathbb{R}$}
\end{align}
where $K$ is the strike price of the option. We use the $\max$ since the option has no value 
if the strike is above the market value. We can then consider when $z=0$ and 
$z \to \infty$, as in \cite{scholesapplication}. When $z=0$ we notice from \ref{riskyasset} 
that $\mathrm{d}S = 0$ so $z = 0$ is constant and
\begin{equation} \label{boundarycondition}
    u(t,0) = 0.
\end{equation}
As $z \to \infty$, it becomes more likely we exercise the option, and the strike price 
becomes less relevant to $u$, so the value of $u(t,Z)$ is equivalent to $z$. Therefore 
we have the boundary condition 
\begin{align} 
    u(t,z) = z && \text{as $z \to \infty$}
\end{align}

We have finally reached the general Black-Scholes inital value boundary problem

\begin{align}
    u_t(t,z) +\frac{1}{2}\sigma^2z^2u_{zz}(t,z) + rzu_z(t,z) - ru(t,z) = 0 &&
    \text{for $(t,z) \in (0,T) \times \mathbb{R}_+ $}\\
    \text{with initial condition: } && u(0,z) = \max{(z-K, 0)} \text{, } 
    z \in \mathbb{R}_+ \\
    \text{and boundary conditions: } && u(t, 0) = 0 \text{,  $u(t,z) = z$ as 
    $S \to \infty $,  $t \in [0,T]$}
\end{align}
Where we remember that: 
\begin{align}
    u(t,z) - && \text{price of the option}\\
    z - && \text{price of the underlying stock}\\
    K - && \text{strike price of the option}\\
    r - && \text{annualized risk-free interest rate, continouusly compounded}\\
    t - && \text{time, generally in years, with now as $t=0$ and expiry $t=T$}\\
    \sigma - && \text{the volatility of the underlying stock}
\end{align}
\section{The Black-Scholes Model}
Now we have derived and fully defined our model, we may consider final and boundary 
conditions of a European Call, as defined in \ref{typesofoption}. We define this as 
$C(t,z)$, with exercise price $K$ and expiry date $T$ from above. Now for the final 
condition $t=T$ we appeal to the definition of a call as in \ref{initialvalue}:
\begin{equation}
    C(T, z) = \max{(z-K, 0)}
\end{equation}

Now for our boundary conditions again we consider $u$ for $z=0$ and $z \to \infty$. Simiarly 
to \ref{boundarycondition} we have 
\begin{align}
    C(t, 0) = 0 &&\text{for } t \in (0,T)
\end{align}
Then for $z \to \infty$, again we can ignore out max. However, in this case, we must 
purchase the stock and hold on to it until the exercise date, as we have a 
european option. This money spent on the stock could otherwise be put in a 
money market account, so the profit from exercising the option must account for this.
From \ref{riskfree solution}, we get that 
\begin{align}
    C(t,z) = z - Ke^{-r(T-t)} &&\text{as } z \to \infty
\end{align}
We then finally get the following, the Black-Scholes European Call option IBVP:

\begin{align}
    u_t(t,z) +\frac{1}{2}\sigma^2z^2u_{zz}(t,z) + rzu_z(t,z) - ru(t,z) = 0 &&
    \text{for $(t,z) \in (0,T) \times \mathbb{R}_+ $}\\
    \text{with initial condition: } &&
     u(0,z) = \max{(z-K, 0)} \text{, } z \in \mathbb{R}_+ \\
    \text{and boundary conditions: } && 
    u(t, 0) = 0 \text{,  $u(t,z) = z - Ke^{-r(T-t)}$}\\
    \text{as $z \to \infty $,  $t \in [0,T]$}
\end{align}

\section{Modelling GameStop}
We begin by using \ref{volatilityapprox} a calculate a value of $\sigma$. Using data from 
\cite{nyse} between 2018 - 2020, in the attached excel book we 
calculate $\sigma = 1.007310782$. We also set 
$r=1.004$, approximately the average bank rate between 2018 - 2020.
\begin{figure}[h] 
    \centering
    \includegraphics[width=0.5\textwidth]{sigmavalue.png} 
    \caption[]{}
    \label{sigmavalue}
\end{figure}
\ref{sigmavalue} shows how returns from options are far more consistent and safe with a low a low volatility,
 like in the GameStop case. In the above case we have used the solution from \cite{wikipedia}, 
 so the upper to sigma is $x>0$ so:
\begin{align}
    x =&& \ln{\frac{z}{K}} + (r - \frac{1}{2} \sigma^2)\tau > 0 && 
    \text{with $\tau$, $x$ as defined in \cite{wikipedia}} \\
    \implies |\sigma| <&& \sqrt{\frac{2}{\tau} \ln{\frac{z}{K} + 2r}} && \approx 1.435
\end{align}
Modifying values of K, the strike price, we see in \ref{kvalue} the increased risk 
of buying options with a more expensive strike price, which makes sense - if 
the strike of an option is near the current price, it is more likely that 
the stock drops below this and so expires worthless.
\begin{figure}[h] 
    \centering
    \includegraphics[width=0.5\textwidth]{kvalue.png} 
    \caption[]{}
    \label{kvalue}
\end{figure}
Again here we have a mathematical limit on the value of k:
\begin{align}
    \ln{\frac{z}{K}} + (r - \frac{1}{2} \sigma^2)\tau > 0 \\
    \implies |k| < ze^{\frac{1}{2}\sigma^2 - r} \approx 12.171166822082068
\end{align}
We finish with an analysis for different values of $\tau$, where we notice that 
the time to expiry has no effect on the distribution of $u$, the value of the 
option.
\begin{figure}[h] 
    \centering
    \includegraphics[width=0.5\textwidth]{tauvalue.png} 
    \caption[]{}
    \label{tauvalue}
\end{figure}
\begin{align}
    \ln{\frac{z}{K}} + (r - \frac{1}{2} \sigma^2)\tau > 0 \\
    \implies |\tau| > \frac{\ln{\frac{K}{z}}}{r - \frac{1}{2}\sigma^2} 
    \approx -0.0510&& \text{although note that time is 
    defined to be nonegative.}
\end{align}
\section{Conclusion}
In conclusion, it is apparent that GameStop was an extremely safe investment 
for these hedge funds - with low volatity and and slow decline, these head 
funds could gain millions in low risk money of the period of many years, which 
is how it ended up gaining such a massive short interest which allowed the 
squeeze to occur so easily.
\printbibliography %Prints bibliography
\end{document}

