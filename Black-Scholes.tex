% !TEX TS-program = pdflatex
% !TEX encoding = UTF-8 Unicode

% This is a simple template for a LaTeX document using the "article" class.
% See "book", "report", "letter" for other types of document.

\documentclass[11pt]{article} % use larger type; default would be 10pt

\usepackage[utf8]{inputenc} % set input encoding (not needed with XeLaTeX)
\usepackage{biblatex} %Imports biblatex package
\addbibresource{Black-Scholes.bib} %Import the bibliography file

%%% Examples of Article customizations
% These packages are optional, depending whether you want the features they provide.
% See the LaTeX Companion or other references for full infofrmation.

%%% PAGE DIMENSIONS
\usepackage{geometry} % to change the page dimensions
\geometry{a4paper} % or letterpaper (US) or a5paper or....
% \geometry{margin=2in} % for example, change the margins to 2 inches all round
% \geometry{landscape} % set up the page for landscape
%   read geometry.pdf for detailed page layout information

\usepackage{graphicx} % support the \includegraphics command and options

% \usepackage[parfill]{parskip} % Activate to begin paragraphs with an empty line rather than an indent

%%% PACKAGES
%\usepackage{hyperef}
\usepackage{booktabs} % for much better looking tables
\usepackage{amsmath}
\usepackage{array} % for better arrays (eg matrices) in maths
\usepackage{paralist} % very flexible & customisable lists (eg. enumerate/itemize, etc.)
\usepackage{verbatim} % adds environment for commenting out blocks of text & for better verbatim
\usepackage{subfig} % make it possible to include more than one captioned figure/table in a single float
% These packages are all incorporated in the memoir class to one degree or another...

%%% HEADERS & FOOTERS
\usepackage{fancyhdr} % This should be set AFTER setting up the page geometry
\pagestyle{fancy} % options: empty , plain , fancy
\renewcommand{\headrulewidth}{0pt} % customise the layout...
\lhead{}\chead{}\rhead{}
\lfoot{}\cfoot{\thepage}\rfoot{}

%%% SECTION TITLE APPEARANCE
\usepackage{sectsty}
\allsectionsfont{\sffamily\mdseries\upshape} % (See the fntguide.pdf for font help)
% (This matches ConTeXt defaults)

%%% ToC (table of contents) APPEARANCE

\usepackage{amsfonts} 
\usepackage[nottoc,notlof,notlot]{tocbibind} % Put the bibliography in the ToC
\usepackage[titles,subfigure]{tocloft} % Alter the style of the Table of Contents
\renewcommand{\cftsecfont}{\rmfamily\mdseries\upshape}
\renewcommand{\cftsecpagefont}{\rmfamily\mdseries\upshape} % No bold!
%%% END Article customizations

\newtheorem{theorem}{Theorem}[section]
\newtheorem{corollary}{Corollary}[theorem]
\newtheorem{lemma}[theorem]{Lemma}


%%% The "real" document content comes below...

\title{GameStop: A study in the Black-Scholes Formula}
\author{Luke Stanislaus 2009701}
%\date{} % Activate to display a given date or no date (if empty),
         % otherwise the current date is printed 

\begin{document}

\maketitle

\begin{abstract}
    This is a simple paragraph at the beginning of the document. A brief introduction to the main subject.
    \end{abstract}

\section{Introduction}


\section{What are shares, options, and short selling?}
\subsection{Shares}
As defined in "Investopedia" \cite{shares}, shares are "units of equity 
ownership in a corporation." Writing shares is an effective way to raise 
capital for a firm, as there is not legal mandate to be repaid to 
investors, and to not pay interest. Common shares usually offer a voting 
rights, giving shareholders more control over a business. The value of 
shares rises and falls with demand - meaning that owning a stock assumes 
the owner a level of risk. Many stocks also offer dividends to 
shareholders, which is a proportion of the businesses profits paid out 
directly.
\subsection{Options}
Again, as defined in \cite{options}, "the term option refers to a 
financial instrument based on the value of the underlying security, 
such as stocks." An options contract offers the holder the option, 
not the obligation (in contrast to a "future"), to buy or sell a share 
at a predetermined price. In themselves, they are a form of asset and 
as such have their own valuations and market value. A "call" option 
is when the writer of the option promises to sell the stock at a 
predetermined price, called the "strike" at a predetermined time, 
called the "strike date". This means that if you own a call option and 
the stock price increases above the strike, the writer will still have 
to honor the strike and so you can "exercise" the option 
and immediately sell the stock, where your profit is the difference 
between the strike and current price, less the cost of the option. 
However, if the price goes below the strike, you would be forced 
to let your option expire worthless, since by exercising the option 
you would pay above the market rate. In this way, the losses are limited 
but the profits are unlimited - you can only lose what you paid for the 
call options, but stand to gain an unlimited amount as the stock price 
increases. 
\paragraph{}
On the other hand, if an investor expects a stock to decrease 
in value, they could write a call (also known as a short call). This 
means that if the stock has increase in value and you don't already own 
the security, you are obligated to purchase at the market rate, which 
could take any arbitrary value. Further to this, as the price increases 
over time, the short seller may be "margin called" - when to ensure the 
seller can pay, the broker requests that the seller puts more money into 
their account. If the short seller cannot afford this (and the further 
risk it entails), they can close their position - by buying the stocks 
they owe, the are protected from any further increases in price. However, 
by buying the stock off the open market they increase demand for the 
stock, increasing the price further, forcing other short sellers to 
close their positions, driving the price up further. This is what is 
known as a short squeeze.

\section{What happened to GameStop?}

\section{Asset dynamics}

\subsection{A risk-free asset}
A risk-free asset, as defined in The Black-Scholes Model \cite{blackscholes} can be thought
of as a money-market account with zero risk, for example a bank account with an interest 
rate. It is described by the deterministic function 
\begin{equation} \label{riskfree}
    dA(t) = rA(t)\mathrm{d}t 
\end{equation}
Where in \ref{riskfree}, $r$ represents the risk-free rate and $A$ the value of the 
asset. We typically let $r>0$ and $A(0) = 1$ for convenience. This can be rewritten as 
the ordinary differential equation $A'(t) = rA(t)$, which has the unique solution 
\begin{equation} \label{riskfree solution}
    A(t) =e^{rt}
\end{equation}
\subsection{Ito processes}

According to Ito process \cite{itoprocess}, an Ito process is the stochastic (random) 
process $X = \{X_t, t>=0\}$ which solves 
\begin{equation} \label{randomprocess}
X_t = X_0 + \int_{0}^{t} a(X_s, s)\mathrm{d}s + 
\int_{0}^{t} b(X_s, s) \mathrm{d}W_s
\end{equation}
Here $X_0$ is the "scalar starting point", and $a,b$ are stochastic 
processes defined for $\{a(X_t, t) : t\geq0\} $ and $\{b(X_t, t) : t\geq0\}$. 
Then we call $a$ the drift of the variable, and $b$ is the diffusion, 
or more precisely in a financial context, volatity. \ref{randomprocess} is commonly 
written as 
\begin{equation}
    X_t = a(X_t, t)\mathrm{d}t + b(X_t, t)\mathrm{d}W_t
\end{equation}
or even 
\begin{equation}
    X_t = a_t\mathrm{d}t + b_t\mathrm{d}W_t
\end{equation}
This converges at some time $T$ to Brownian motion with instantaneous drift $a_T$ and
variance $b_T^2$. We let $dW$ be normally distributed with mean zero and variance 
$\mathrm{d}t$. Then we can further simplify \ref{randomprocess} as 
\begin{equation}
    \mathrm{d}X_t = a_t\mathrm{d}t + b_t \sqrt{\mathrm{d}t}\xi
\end{equation}
where
\begin{equation}
    \xi \sim \mathcal{N}(\mu,\,\sigma^{2})
\end{equation}

As defined in \cite{itoprocess}, Ito's lemma is 

\begin{lemma}[Ito's Lemma]
    Suppose $f: R \to R$ is twice continuously differentiable and 
    $\mathrm{d}X = a_t\mathrm{d}t + b_t\mathrm{d}W$. Then $f(X)$ is the Ito process,
    \begin{equation}
        f(X_t)
        = f(X_0) + \int_0^t \! f'(X_s)a_s \, \mathrm{d}s + 
        \int_0^t \! f'(X_s)b_s \, \mathrm{d}W + \frac{1}{2}\int_0^t \! f''(X_s)b_s^2 \,
         \mathrm{d}s
    \end{equation}
    for $t\ge0$
\end{lemma}
We shall also require Ito's existence/uniqueness theorem, as defined in 
\cite{SDE}:

\begin{theorem}[Ito's Existence And Uniqueness Theorem] 
    \label{isosexistence}
    If $\mu : \mathbb{R} \to \mathbb{R}$ and $\sigma : \mathbb{R} \to \mathbb{R}_+$ 
    are uniformally Lipschitz, then the stochastic differential equation 
    \ref{randomprocess} has "strong solutions". This means that for any standard 
    Brownian motion $\{W_t\}_{t\geq0}$, any admissable filtration $\mathbb{F} = 
    \{\mathcal{F}_t\}$ and any initial value $x \in \mathbb{R}$ there exists a 
    unique proccess $X_t = X_t^x$ which solves \ref{randomprocess}.

    
\end{theorem}

\subsection{A risky asset}

A risk asset, as defined in The Black-Scholes Model \cite{blackscholes}, 
can be thought of as a stock, is represented as the Ito process

\begin{equation} \label{riskyasset}
    \mathrm{d}S(t) = \mu S(t)\mathrm{d}t + \sigma S(t) \mathrm{d}W(t)
\end{equation}

with $S(0)$ a given starting price of the stock, $\mu \in \mathbb{R}$ 
is again the drift and $\sigma > 0$ is the volatility of the stock 
price $S$. Using \ref{isosexistence} we can prove existence and uniqueness of 
\ref{riskyasset}:

First we rewrite \ref{riskyasset} in integral form:
\begin{equation}\label{riskyint}
    S(t) = S(0) + \mu\int_0^t \! S(u) \, \mathrm{d}u + \sigma\int_0^t \! S(u) \, 
    \mathrm{d}W(u).
\end{equation}

Then we rewrite our coefficients in the form of \ref{randomprocess}: 

\begin{align}
    \mu S(t) = a(t,S(t)), && a(t,x) = \mu x \\
    \sigma S(t) = b(t,S(t), && b(t,x) = \sigma x
\end{align}

then we check Lipschitz continuity:

\begin{align}
    |a(t,x) - a(t,y)| &= |\mu(x-y)| \leq |\mu||x-y|, \\
    |b(t,x) - b(t,y)| &= |\sigma(x-y)| \leq \sigma|x-y| &&\text{Since $\sigma > 0$.}
\end{align}

We then have that the unique solution to \ref{riskyasset} is of the form 
\begin{equation}
    S(t) = S(0)exp\{\mu t - \frac{\sigma^2}{2}t + \sigma W(t)\}.
\end{equation}
This can be easily substituted into \ref{riskyint} show that this solves \ref{riskyasset}.

\subsection{Considering the model parameters}

Now we have a soluition to a risky asset, we can consider $\mathbb{E}(S(t))$:

\begin{align}
    \mathbb{E}(S(t))  = &&S(t)\mathbb{E}(exp\{\mu - \frac{1}{2}\sigma t + \sigma W(t)\})\\
     \label{variance}= 
     && S(0)exp\{\mu t - \frac{1}{2}\sigma^2 t\}\mathbb{E}(exp\{\sigma W(t)\} \\
    \label{result} = && S(0)exp\{\mu t\}).
\end{align}

Where at \ref{variance} we use the fact that 

\begin{equation*}
    \mathbb{E}(exp\{X\}) = exp\{\frac{1}{2}Var(X)\}
\end{equation*}

\ref{result} then shows that if $\mu = 0$ we know that the expectation of S(t) is 
constant in time - it is not "drifting" in any direction, which justifies the 
naming of $\mu$ as drift. We can then rearrange for $\mu$:

\begin{equation} \label{mu}
    \mu = \frac{1}{t}\ln{\frac{\mathbb{E}(S(t))}{S(0)}}.
\end{equation}

which is that logarithmic return of the expected price.
The variance of the return is 
\begin{align}
    Var(\mu t - \frac{\sigma^2}{2}t + \sigma W(t)) = && Var(\sigma W(t)) \\
    = && \sigma^2t  && \text{since Var(W(t)) = t}
\end{align}
as defined in \cite{blackscholes}. Clearly we want to try to calculate $\mu$ and 
$\sigma$ to apply to our model. \ref{mu} could suggest taking past values of the 
stock value and taking an average to find a value of $\mu$, but according to 
\cite{blackscholes} the accuracy of this is poor. 
A more effective approximation of the volatility is, for example, the process 

\begin{equation}
    \ln{S(t)} = \ln{S(0)} + (\mu - \frac{1}{2}\sigma^2)t + \sigma W(t))
\end{equation}

which is an Ito process with the constant characteristics $a(t) = (\mu - \frac{1}{2}\sigma^2)$
 and $b(t) = \sigma$. By \cite{quadtraticvariation}, we have that
 \begin{equation}
     (\mathrm{d}S(t))^2 = b^2(t) \mathrm{d}t
 \end{equation}
 If we then partition $[0,t]$ given by $0 = t_1 < \dots < t_n = t$ with small mesh max 
 width $[t_{k+1} = t_k]$, we have 
 \begin{equation}
     (\ln{S(t_{k+1}) - \ln{S(t_k)}})^2 \approx \sigma^2 \mathrm{d}t
 \end{equation}
then 
\begin{align}
    \sum_k (\ln{S(t_{k+1}) - \ln{S(t_k)}})^2 \approx \sigma^2 t\\
    \implies
    \sigma = \sqrt{\frac{1}{2}\sum_k (\ln{\frac{S(t_{k+1})}{S(t_k)}})^2}
\end{align}
which is a much better estimate of the volatility coefficient, and is described as the 
sample volatility.
\section{Options}

\section{The Black-Scholes Model}

\section{Modelling GameStop}

\section{Extensions to the Black-Scholes Model}

\section{Conclusion}

\printbibliography %Prints bibliography
\end{document}
